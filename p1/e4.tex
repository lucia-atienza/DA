El algoritmo usa dos estructuras para manejar los datos, una matriz y una lista. Para la matriz se ha usado
el contenedor de la STL std::vector. Se llama a la funcion rellenaMapa para asignar valores a todas y cada una
de las celdas del mapa. Luego se han almacenado las celdas en una lista para poder ordenarlas por valor,
permitiendo que la funci\'on extraeCelda devuelva la mejor celda de todas las disponibles en coste constante.
El algoritmo sigue el siguiente esquema de los algoritmos devoradores:\\
$ listaCeldas \leftarrow rellenarCon(mapa) $ \\
listaCeldas.sort()\\
centroColocado $\leftarrow$ false \\
mientras $ \neg  centroColocado  \land listaCeldas  \neq \emptyset $\\
\hspace*{1cm} p $\leftarrow extraccionCelda(listaCeldas)$\\
\hspace*{1cm} si factible(p)\\
\hspace*{2cm} centroColocado  $\leftarrow$ true\\

Se distinguen los siguientes elementos caracter\'isticos de los algoritmos devoradores:
\begin{itemize}
    \item Conjunto de candidatos: las celdas disponibles.   
    \item Candidatos seleccionados: la primera celda en la que sea factible colocar el centro de extracción de minerales.
    \item Funcion de factibilidad: la funci\'on factible.
    \item Funci\'on de selecci\'on: la funci\'on extraeCelda().
    \item Soluci\'on: colocar el centro de extracci\'on de minerales.
    \item Objetivo: colocar el centro extracci\'on de minerales en la mejor celda para maximizar el tiempo de vida del mismo.
\end{itemize}
