\documentclass[]{article}

\usepackage[left=2.00cm, right=2.00cm, top=2.00cm, bottom=2.00cm]{geometry}
\usepackage[spanish,es-noshorthands]{babel}
\usepackage[utf8]{inputenc} % para tildes y ñ
\usepackage{graphicx} % para las figuras
\usepackage{xcolor}
\usepackage{listings} % para el código fuente en c++

\lstdefinestyle{customc}{
  belowcaptionskip=1\baselineskip,
  breaklines=true,
  frame=single,
  xleftmargin=\parindent,
  language=C++,
  showstringspaces=false,
  basicstyle=\footnotesize\ttfamily,
  keywordstyle=\bfseries\color{green!40!black},
  commentstyle=\itshape\color{gray!40!gray},
  identifierstyle=\color{black},
  stringstyle=\color{orange},
}
\lstset{style=customc}

%opening
\title{Práctica 4. Exploración de grafos}
\author{Lucía Atienza Olmo \\ % mantenga las dos barras al final de la línea y este comentario
lucia.atienzaolmo@alum.uca.es \\ % mantenga las dos barras al final de la línea y este comentario
Teléfono: 646767043 \\ % mantenga las dos barras al final de la linea y este comentario
NIF: 32089267n  \\ % mantenga las dos barras al final de la línea y este comentario
}


\begin{document}

\maketitle

%\begin{abstract}
%\end{abstract}

% Ejemplo de ecuación a trozos
%
%$f(i,j)=\left\{ 
%  \begin{array}{lcr}
%      i + j & si & i < j \\ % caso 1
%      i + 7 & si & i = 1 \\ % caso 2
%      2 & si & i \geq j     % caso 3
%  \end{array}
%\right.$

\begin{enumerate}
\item Comente el funcionamiento del algoritmo y describa las estructuras necesarias para llevar a cabo su implementación.

$$ f(rango, coste, dispersion, danno)=2 * (rango - danno - 2*dispersion) + 1.7 * (danno - 0.2*coste) $$


\item Incluya a continuación el código fuente relevante del algoritmo.

\begin{lstlisting}
void DEF_LIB_EXPORTED calculatePath(AStarNode* originNode, AStarNode* targetNode
                   , int cellsWidth, int cellsHeight, float mapWidth, float mapHeight
                   , float** additionalCost, std::list<Vector3> &path) 
{
    bool found = false;
    std::vector<AStarNode*> closed, opened;
    AStarNode* cur = originNode;
    int x, y;
    positionToCell((cur)->position, x, y, cellsWidth, cellsHeight);
    cur->H = estimatedDistance(originNode, targetNode, x, y, additionalCost);
    cur->G = 0; 
    cur->F = cur->G + cur->H;
    opened.push_back(cur);
    std::make_heap(opened.begin(), opened.end(), min);
    while(!found && opened.size() > 0) //mientras no se encuentre solucion y queden nodos disponibles
    {
        std::pop_heap(opened.begin(), opened.end(), min);
        cur = opened.back();
        opened.pop_back(); 
        closed.push_back(cur);
        if(iguales(cur, targetNode))
            found = true;
        else
        {
            List<AStarNode*>::iterator j; 
            for(j = cur->adjacents.begin(); j != cur->adjacents.end(); ++j) 
            {
                if(std::find_if(closed.begin(), closed.end(), isEqualTo(*j)) == std::end(closed)) //no se encuentra en cerrados
                {
                    if(std::find_if(opened.begin(), opened.end(), isEqualTo(*j)) == std::end(opened)) //no se encuentra en abiertos
                    {
                        (*j)->parent = cur;
                        (*j)->G = cur->G + _distance(cur->position, (*j)->position);
                        int row,col;
                        positionToCell((*j)->position, row, col, cellsWidth, cellsHeight);
                        (*j)->H = estimatedDistance(*j, targetNode, row, col, additionalCost);
                        (*j)->F = (*j)->G + (*j)->H;
                        opened.push_back(*j);
                        std::push_heap(opened.begin(), opened.end(), min);
                    }
                    else //se ha encontrado en abiertos
                    {
                        float d = _distance(cur->position, (*j)->position);
                        if((*j)->G > cur->G + d)
                        {
                            (*j)->parent = cur;
                            (*j)->G = cur->G + d;
                            (*j)->F = (*j)->G + (*j)->H;
                            std::sort_heap(opened.begin(), opened.end(), min);
                        }
                    }
                }
            }
        }
    }
    if(found) //si hemos encontrado solucion, recuperamos camino
    {
        cur = targetNode;
        path.push_front(targetNode->position);
		while(cur->parent != originNode)
		{
			cur = cur->parent;
			path.push_front(cur->position);
		}
    }
}
\end{lstlisting}


\end{enumerate}

Todo el material incluido en esta memoria y en los ficheros asociados es de mi autoría o ha sido facilitado por los profesores de la asignatura. Haciendo entrega de esta práctica confirmo que he leído la normativa de la asignatura, incluido el punto que respecta al uso de material no original.

\end{document}
